\chapter{集合和逻辑用语}
\setcounter{choicecounter}{0} 

\subsection*{A组}

\begin{choice}{}{量词命题的否定}
	已知命题 $p: \exists x \in \mathbb{N}, x^2 \le 0$,则 $\neg p$ 为
	\begin{tasks}(2)
		\task $\exists x \in \mathbb{N}, x^2 \le 0$
		\task $\exists x \in \mathbb{N}, x^2 > 0$
		\task $\forall x \in \mathbb{N}, x^2 > 0$
		\task $\forall x \in \mathbb{N}, x^2 \ge 0$
	\end{tasks}
\end{choice}

\begin{choice}{}{集合的补集与交集}
	{(2023 秋 • 高一 • 重庆永川区 • 月考校考)} 设集合 $A=\{x|x<2 \text{或} x \ge 4\}, B=\{x|x<a\}$,若 $(\complement_U A) \cap B \neq \emptyset$,则 $a$ 的取值范围是
	\begin{tasks}(2)
		\task $a < 2$
		\task $a > 2$
		\task $a \le 4$
		\task $a \ge 4$
	\end{tasks}
\end{choice}

\begin{choice}{}{集合的交集与子集个数}
	已知集合 $A=\{(x,y) | x,y \in \mathbb{Z}, \text{且 } xy=4\}, B=\{(x,y) | x \le y\}$,则 $A \cap B$ 的子集的个数为
	\begin{tasks}(4)
		\task 3
		\task 4
		\task 8
		\task 16
	\end{tasks}
\end{choice}


\subsection*{B组}

\begin{choice}{}{充分必要条件}
	{(2023 秋 • 高一 • 山东烟台 • 月考校考)} 下列说法错误的是
	\begin{tasks}(1)
		\task “$A \cap B = B$” 是 “$B = \emptyset$” 的必要不充分条件
		\task “$x=3$” 的一个充分不必要条件是 “$x^2-2x-3=0$”
		\task “$|x|=1$” 是 “$x=1$” 的必要不充分条件
		\task “$m$ 是实数” 的一个充分不必要条件是 “$m$ 是有理数”
	\end{tasks}
\end{choice}

\begin{choice}{}{全称命题的否定}
	若命题 “$\forall x \in \mathbb{R}, 1-x^2 \le m$” 是假命题,则实数 $m$ 的取值范围是
	\begin{tasks}(2)
		\task $(-\infty, 1)$
		\task $(-\infty, 1]$
		\task $(1, +\infty)$
		\task $[1, +\infty)$
	\end{tasks}
\end{choice}

\begin{choice}{}{充分不必要条件的应用}
	若 “$x=2$” 是 “$m^2x^2 - (m+3)x + 4 = 0$” 的充分不必要条件,则实数 $m$ 的值为
	\begin{tasks}(2)
		\task 1
		\task $-\frac{1}{2}$
		\task $-\frac{1}{2}$ 或 1
		\task -1 或 $-\frac{1}{2}$
	\end{tasks}
\end{choice}

\begin{choice}{}{集合的补集与互异性}
	设全集 $U=\{1,2,3,4,5\}$,集合 $A=\{1,a,b\}, B=\{4, a-b\}$.若 $\complement_U A = B$,则 $a,b$ 的值分别为
	\begin{tasks}(2)
		\task 3,2
		\task 4,3
		\task 3,2 或 5,3
		\task 5,2 或 5,3
	\end{tasks}
\end{choice}


\subsection*{C组}

\begin{choice}{}{新定义集合与元素互异性}
	{(2024 秋 • 高一 • 福建龙岩 • 开学考试校考)} 定义集合 $A \otimes B = \{x | x=\sqrt{a^2+b^2}, a \in A, b \in B\}$,若 $A=\{n, -1\}, B=\{\sqrt{2}, 1\}$,且集合 $A \otimes B$ 中有 3 个元素,则由实数 $n$ 的所有取值组成的集合的非空真子集的个数为
	\begin{tasks}(4)
		\task 2
		\task 6
		\task 14
		\task 15
	\end{tasks}
\end{choice}

\subsection*{德摩根定律与容斥原理}

\subsubsection*{A组}

\begin{exercise}{5}{集合的并集与补集运算}
	\textbf{(填空题)} 已知全集 $U=\{x \in \mathbb{Z} \mid -3 \le x \le 3\}$, 集合 $A=\{-2, 1, 2\}$, $B=\{-1, 0, 1\}$. 求 $\complement_U(A \cup B) = $ \rule{3cm}{0.5pt}.
\end{exercise}

\begin{exercise}{10}{容斥原理初步应用}
	\textbf{(解答题)} 某班有50名学生,参加数学兴趣小组的有25人,参加物理兴趣小组的有22人,两个小组都参加的有10人.问:
	\begin{enumerate}
		\item 至少参加一个兴趣小组的学生有多少人?
		\item 两个小组都没有参加的学生有多少人?
	\end{enumerate}
\end{exercise}

\subsubsection*{B组}

\begin{exercise}{12}{德摩根定律与一元二次不等式}
	\textbf{(解答题)} 设全集 $U=\mathbb{R}$, 集合 $A=\{x|x \le -1 \text{ 或 } x \ge 4\}$, $B=\{x| x^2-3x-4<0\}$.
	\begin{enumerate}
		\item 求集合 $B$;
		\item 利用德摩根定律,求 $\complement_U(A \cap B)$.
	\end{enumerate}
\end{exercise}

\begin{exercise}{12}{容斥原理的逆向应用}
	\textbf{(解答题)} 某次考试中,50名学生参加了数学和物理两科.已知数学成绩及格的有40人,物理成绩及格的有31人,两科都不及格的有4人.问这次考试中:
	\begin{enumerate}
		\item 两科都及格的学生有多少人?
		\item 恰好只有一科及格的学生有多少人?
	\end{enumerate}
\end{exercise}


\subsubsection*{C组}

\begin{exercise}{12}{三集合容斥原理}
	\textbf{(解答题)} 某新闻机构就A, B, C三个热点话题对100人进行调查,结果显示:关注A的有45人,B的有55人,C的有60人;同时关注A和B的有25人,A和C的有20人,B和C的有30人;三个话题都关注的有10人.问:
	\begin{enumerate}
		\item 至少关注一个话题的人数是多少?
		\item 恰好只关注一个话题的人数是多少?
	\end{enumerate}
\end{exercise}