% !TEX program = xelatex
\documentclass[12pt,a4paper]{ctexbook}
\usepackage[left=2cm, right=2cm, top=3cm, bottom=3cm]{geometry}

\usepackage{amsmath}
\usepackage{amssymb}
\usepackage{amsthm}
\usepackage[colorlinks=true, linkcolor=black, citecolor=black]{hyperref}

\theoremstyle{definition}
\newtheorem{exercise}{习题} 
\theoremstyle{remark}
\newtheorem*{solution}{解析} 

\begin{document}
	\begin{titlepage}
		\thispagestyle{empty} 
		\begin{flushleft}
			\vspace*{2cm}
			{\large
				知乎@余命数 \\
				Midnight Forever
			}
			\vspace{4cm}
			{\fontsize{36}{44}\selectfont\bfseries 高中数学之旅·习题集}\\[2cm]
			
			{\fontsize{22}{28}\selectfont An Excursion through Secondary School Mathematics, Exercise Book}

			\vspace{2cm}

			\vfill

			{\small
				最后更新: \today \\
				书稿版本
			}
		\end{flushleft}
	\end{titlepage}
	
	\frontmatter
	\pagestyle{plain}
	\tableofcontents
	
	\mainmatter
	\chapter{集合和逻辑用语}
	
	\subsection*{A组}
	\begin{enumerate}
		\item 已知命题 $p: \exists x \in \mathbb{N}, x^2 \le 0$,则 $\neg p$ 为 ( \hspace{2cm} )
		\begin{enumerate}
			\item $\exists x \in \mathbb{N}, x^2 \le 0$
			\item $\exists x \in \mathbb{N}, x^2 > 0$
			\item $\forall x \in \mathbb{N}, x^2 > 0$
			\item $\forall x \in \mathbb{N}, x^2 \ge 0$
		\end{enumerate}
		
		\item {[2023 秋 • 高一 • 重庆永川区 • 月考校考]} 设集合 $A=\{x|x<2 \text{或} x \ge 4\}, B=\{x|x<a\}$,若 $(∁_U A) \cap B \neq \emptyset$,则 $a$ 的取值范围是 ( \hspace{2cm} )
		\begin{enumerate}
			\item $a < 2$
			\item $a > 2$
			\item $a \le 4$
			\item $a \ge 4$
		\end{enumerate}
		
		\item 已知集合 $A=\{(x,y) | x,y \in \mathbb{Z}, \text{且 } xy=4\}, B=\{(x,y) | x \le y\}$,则 $A \cap B$ 的子集的个数为 ( \hspace{2cm} )
		\begin{enumerate}
			\item 3
			\item 4
			\item 8
			\item 16
		\end{enumerate}
	\end{enumerate}
	
	\subsection*{B组}
	\begin{enumerate}
		\setcounter{enumi}{3}
		\item {[2023 秋 • 高一 • 山东烟台 • 月考校考]} 下列说法错误的是 ( \hspace{2cm} )
		\begin{enumerate}
			\item “$A \cap B = B$” 是 “$B = \emptyset$” 的必要不充分条件
			\item “$x=3$” 的一个充分不必要条件是 “$x^2-2x-3=0$”
			\item “$|x|=1$” 是 “$x=1$” 的必要不充分条件
			\item “$m$ 是实数” 的一个充分不必要条件是 “$m$ 是有理数”
		\end{enumerate}
		
		\item 若命题 “$\forall x \in \mathbb{R}, 1-x^2 \le m$” 是假命题,则实数 $m$ 的取值范围是 ( \hspace{2cm} )
		\begin{enumerate}
			\item $(-\infty, 1)$
			\item $(-\infty, 1]$
			\item $(1, +\infty)$
			\item $[1, +\infty)$
		\end{enumerate}
		
		\item 若 “$x=2$” 是 “$m^2x^2 - (m+3)x + 4 = 0$” 的充分不必要条件,则实数 $m$ 的值为 ( \hspace{2cm} )
		\begin{enumerate}
			\item 1
			\item $-\frac{1}{2}$
			\item $-\frac{1}{2}$ 或 1
			\item -1 或 $-\frac{1}{2}$
		\end{enumerate}
		
		\item 设全集 $U=\{1,2,3,4,5\}$,集合 $A=\{1,a,b\}, B=\{4, a-b\}$.若 $∁_U A = B$,则 $a,b$ 的值分别为 ( \hspace{2cm} )
		\begin{enumerate}
			\item 3,2
			\item 4,3
			\item 3,2 或 5,3
			\item 5,2 或 5,3
		\end{enumerate}
	\end{enumerate}
	
	\subsection*{C组}
	\begin{enumerate}
		\setcounter{enumi}{7}
		\item {[2024 秋 • 高一 • 福建龙岩 • 开学考试校考]} 定义集合 $A \otimes B = \{x | x=\sqrt{a^2+b^2}, a \in A, b \in B\}$,若 $A=\{n, -1\}, B=\{\sqrt{2}, 1\}$,且集合 $A \otimes B$ 中有 3 个元素,则由实数 $n$ 的所有取值组成的集合的非空真子集的个数为 ( \hspace{2cm} )
		\begin{enumerate}
			\item 2
			\item 6
			\item 14
			\item 15
		\end{enumerate}
	\end{enumerate}
	
	\subsection*{德摩根定律与容斥原理}
	
	\subsubsection*{A组}
	\begin{enumerate}
		\item \textbf{(填空题)} 已知全集 $U=\{x \in \mathbb{Z} \mid -3 \le x \le 3\}$, 集合 $A=\{-2, 1, 2\}$, $B=\{-1, 0, 1\}$. 求 $\complement_U(A \cup B) = $ \rule{3cm}{0.5pt}.
		
		\item \textbf{(解答题)} 某班有50名学生,参加数学兴趣小组的有25人,参加物理兴趣小组的有22人,两个小组都参加的有10人.问:
		\begin{enumerate}
			\item 至少参加一个兴趣小组的学生有多少人?
			\item 两个小组都没有参加的学生有多少人?
		\end{enumerate}
	\end{enumerate}
	
	\subsubsection*{B组}
	\begin{enumerate}
		\setcounter{enumi}{2}
		\item \textbf{(解答题)} 设全集 $U=\mathbb{R}$, 集合 $A=\{x|x \le -1 \text{ 或 } x \ge 4\}$, $B=\{x| x^2-3x-4<0\}$.
		\begin{enumerate}
			\item 求集合 $B$;
			\item 利用德摩根定律,求 $\complement_U(A \cap B)$.
		\end{enumerate}
		
		\item \textbf{(解答题)} 某次考试中,50名学生参加了数学和物理两科.已知数学成绩及格的有40人,物理成绩及格的有31人,两科都不及格的有4人.问这次考试中:
		\begin{enumerate}
			\item 两科都及格的学生有多少人?
			\item 恰好只有一科及格的学生有多少人?
		\end{enumerate}
	\end{enumerate}
	
	\subsubsection*{C组}
	\begin{enumerate}
		\setcounter{enumi}{4}
		\item \textbf{(解答题)} 某新闻机构就A, B, C三个热点话题对100人进行调查,结果显示:关注A的有45人,B的有55人,C的有60人;同时关注A和B的有25人,A和C的有20人,B和C的有30人;三个话题都关注的有10人.问:
		\begin{enumerate}
			\item 至少关注一个话题的人数是多少?
			\item 恰好只关注一个话题的人数是多少?
		\end{enumerate}
	\end{enumerate}
	
	\chapter{解三角形}
	
	\section{取值范围}
	\subsection{知识回顾}
	解决问题的第一步,是将题目中多变量换成单变量. 这是化繁为简、确立解题方向的关键.
	\begin{itemize}
		\item \textbf{若表达式含边或边的比例},优先考虑使用\textbf{正弦定理} ($a=2R\sin A, b=2R\sin B, \dots$),将边长转化为对应角的正弦,实现“\textbf{边化角}”.
		\item \textbf{若表达式含边的平方},或已知条件为“两边一夹角” (SAS) / “三边” (SSS),则\textbf{余弦定理}是构造关系式的不二之选. 
		\item \textbf{利用内角和定理} $A+B+C=\pi$ 消去多余的角变量,是实现“变量化归”的最后一步.
	\end{itemize}
	
	牢记定义域,这是最关键也最容易被忽略的一步. 三角形中的角并非任意取值,其范围受到严格的内在约束. 忘记定义域,犹如脱缰之马,所得结论往往谬以千里.
	\begin{itemize}
		\item \textbf{基本约束}:三角形的任意内角都必须在 $(0, \pi)$ 范围内.
		\item \textbf{结构约束}:例如,若已将其他角用变量角 $B$ 表示为 $A=\frac{\pi}{2}-2B, C=\frac{\pi}{2}+B$,则必须同时满足 $B>0, A>0, C>0$,即
		$\begin{cases} B>0 \\ \frac{\pi}{2}-2B>0 \\ \frac{\pi}{2}+B>0 \end{cases}$
		解出这些不等式组,才能得到 $B$ 的最终有效范围.
		\item \textbf{附加约束}:题目若明确指出是“锐角三角形”、“钝角三角形”等,则需要增加所有角小于 $\frac{\pi}{2}$ 或某个角大于 $\frac{\pi}{2}$ 的限制条件.
	\end{itemize}
	
	完成前两步后,问题就从一个复杂的几何问题,转化为了我们极为熟悉的、求函数在特定区间上最值的常规问题. 此时,我们有多种工具可供选择.
	\begin{itemize}
		\item \textbf{辅助角公式}:对于形如 $y=a\sin x + b\cos x$ 的函数,此法是首选.
		\item \textbf{二次函数配方法}:对于形如 $y=a\sin^2 x + b\sin x + c$ 的函数,可通过换元法转化为二次函数在闭区间上的最值问题.
		\item \textbf{求导}:对于更复杂的三角函数,求导是判断其单调性、寻找极值点的最通用、最强大的方法.
	\end{itemize}
	\subsection{习题集}
	\subsubsection{取值范围}
	\begin{exercise}[2022 年全国新高考 I 卷第 18 题]\label{ex:gaokao2022_18}
		记 $\triangle ABC$ 的内角 $A,B,C$ 的对边分别为 $a,b,c$, 已知 $\frac{\cos A}{1+\sin A} = \frac{\sin 2B}{1+\cos 2B}$.
		\begin{enumerate}
			\item[(1)] 若 $C = \frac{2\pi}{3}$, 求 $B$;
			\item[(2)] 求 $\frac{a^2+b^2}{c^2}$ 的最小值.
		\end{enumerate}
	\end{exercise}
	
	\begin{exercise}[2023 年衡水中学第三次综合素养评价第 18 题]\label{ex:hengshui2023_18}
		已知在 $\triangle ABC$ 的内角 $A,B,C$ 所对边分别为 $a,b,c$, 且 $a=b+2b\cos C$.
		\begin{enumerate}
			\item[(1)] 求证: $C=2B$;
			\item[(2)] 求 $\frac{a+c}{b}$ 的取值范围.
		\end{enumerate}
	\end{exercise}
	
	\begin{exercise}\label{ex:incenter_area}
		已知 $\triangle ABC$ 的面积为 $S$, 角 $A,B,C$ 所对的边分别为 $a,b,c$.点 $O$ 为 $\triangle ABC$ 的内心, $b = 2\sqrt{3}$ 且 $S = \frac{\sqrt{3}}{4}(a^2+c^2-b^2)$.
		\begin{enumerate}
			\item[(1)] 求 $B$ 的大小;
			\item[(2)] 求 $\triangle AOC$ 的周长的取值范围.
		\end{enumerate}
	\end{exercise}
	
	\begin{exercise}\label{ex:acute_triangle_perimeter}
		在锐角 $\triangle ABC$ 中, 角 $A, B, C$ 所对应的边分别为 $a, b, c$, 已知 $\frac{\sin A - \sin B}{\sqrt{3}a - c} = \frac{\sin C}{a+b}$.
		\begin{enumerate}
			\item[(1)] 求角 $B$ 的值;
			\item[(2)] 若 $a=2$, 求 $\triangle ABC$ 的周长的取值范围.
		\end{enumerate}
	\end{exercise}
	
	\subsubsection{面积问题}
	\begin{exercise}[2023 浙江省十校联盟第三次联考第 18 题]\label{ex:zhejiang2023_18}
		在 $\triangle ABC$ 中, $D$ 为边 $BC$ 上一点, $DC = 3, AD = 5, AC = 7, \angle DAC = \angle ABC$.
		\begin{enumerate}
			\item[(1)] 求 $\angle ADC$ 的大小;
			\item[(2)] 求 $\triangle ABC$ 的面积.
		\end{enumerate}
		\end{exercise}
	% 可以在此处继续添加其他习题,例如:
	% \begin{exercise}[另一道题目]
		%   \label{ex:another_problem}
		%   ...题目内容...
		% \end{exercise}
	
	\chapter*{参考答案}
	\section{集合和逻辑用语}
	
	\subsection*{A组}
	
	\subsubsection*{第 1 题}
	\begin{solution}
		\textbf{【答案】C}
		\paragraph{【解析】}
		本题考查全称量词命题与存在量词命题的否定关系.否定一个量词命题,遵循“改量词,否结论”的原则.
		原命题 $p$ 是一个存在量词命题,其形式为 $\exists x \in M, P(x)$.
		量词 $\exists$ (存在) 的否定是 $\forall$ (任意).
		原结论 $P(x)$ 为 $x^2 \le 0$. 其否定 $\neg P(x)$ 为 $x^2 > 0$.
		故 $\neg p$ 为 $\forall x \in \mathbb{N}, x^2 > 0$.
	\end{solution}
	
	注意:选项 D 中,$x^2 \ge 0$ 是恒成立的,但它不是 $x^2 \le 0$ 的否定.一个结论的否定是其补集,而非简单的反义词.对于实数 $y$,$y \le 0$ 的否定是 $y>0$,而不是 $y \ge 0$.
	
	\subsubsection*{第 2 题}
	\begin{solution}
		\textbf{【答案】B}
		\paragraph{【解析】}
		本题的核心是将集合的抽象运算关系,通过数轴转化为直观的位置关系.关键在于先求出 $∁_U A$.
		集合 $A = \{x|x<2 \text{ 或 } x \ge 4\} = (-\infty, 2) \cup [4, +\infty)$.
		其在全集 $U=\mathbb{R}$ 下的补集为 $A$ 在数轴上未覆盖的部分.
		$∁_U A = \{x|2 \le x < 4\} = [2, 4)$.
		集合 $B = \{x|x<a\} = (-\infty, a)$.
		条件为 $(∁_U A) \cap B \neq \emptyset$,即 $[2, 4)$ 与 $(-\infty, a)$ 的交集不为空.
		为了使两个区间有公共部分,区间 $(-\infty, a)$ 的右端点 $a$ 必须大于区间 $[2,4)$ 的左端点 $2$.
		即 $a > 2$.
	\end{solution}
	
	\subsubsection*{第 3 题}
	\begin{solution}
		\textbf{【答案】D}
		\paragraph{【解析】}
		解题分为两步:第一步,根据集合定义,准确找出交集中的所有元素;第二步,利用公式 $2^n$ 计算子集个数.
		首先,我们用列举法找出集合 $A$ 的所有元素.
		$A$ 的元素是满足 $xy=4$ 的整数坐标点 $(x,y)$.
		$A = \{(1,4), (4,1), (-1,-4), (-4,-1), (2,2), (-2,-2)\}$.
		然后,我们从集合 $A$ 中筛选出满足集合 $B$ 条件 ($x \le y$) 的元素,构成 $A \cap B$.
		$A \cap B = \{(1,4), (-4,-1), (2,2), (-2,-2)\}$.
		集合 $A \cap B$ 中共有 4 个元素.
		一个含有 $n$ 个元素的集合,其子集的个数为 $2^n$.
		本题中 $n=4$,所以子集的个数为 $2^4 = 16$.
	\end{solution}
	
	\subsection*{B组}
	
	\subsubsection*{第 4 题}
	\begin{solution}
		\textbf{【答案】D}
		\paragraph{【解析】}
		本题考查对充分、必要条件的理解.核心是将命题间的逻辑关系转化为集合间的包含关系来判断,即若 $p \Rightarrow q$,则 $p$ 对应的集合是 $q$ 对应集合的子集.
		A. “$A \cap B = B$” $\iff B \subseteq A$. “$B=\emptyset$” $\Rightarrow B \subseteq A$;但 $B \subseteq A \not\Rightarrow B=\emptyset$. 所以 $B \subseteq A$ 是 $B=\emptyset$ 的必要不充分条件.A 正确.
		B. “$x=3$” $\iff x \in \{3\}$. “$x^2-2x-3=0$” $\iff x \in \{3, -1\}$. 因为 $\{3\} \subsetneq \{3, -1\}$,所以前者是后者的充分不必要条件.B 正确.
		C. “$|x|=1$” $\iff x \in \{1, -1\}$. “$x=1$” $\iff x \in \{1\}$. 因为 $\{1\} \subsetneq \{1, -1\}$,所以前者是后者的必要不充分条件.C 正确.
		D. $p$: “$m$ 是有理数”, $q$: “$m$ 是实数”. 有理数集 $\mathbb{Q}$ 是实数集 $\mathbb{R}$ 的真子集,即 $\mathbb{Q} \subsetneq \mathbb{R}$. 因此 $p \Rightarrow q$ 但 $q \not\Rightarrow p$. 所以 $p$ 是 $q$ 的充分不必要条件.原说法称 $q$ 是 $p$ 的充分不必要条件,与事实相反.D 错误.
	\end{solution}
	
	\subsubsection*{第 5 题}
	\begin{solution}
		\textbf{【答案】A}
		\paragraph{【解析】}
		一个全称命题为假,意味着其否定(一个存在量词命题)为真.这是解决此类问题的标准“翻译”步骤.
		设原命题为 $P: \forall x \in \mathbb{R}, 1-x^2 \le m$.
		因为命题 $P$ 是假命题,所以其否定 $\neg P$ 是真命题.
		$\neg P: \exists x \in \mathbb{R}, \neg(1-x^2 \le m)$,即 $\neg P: \exists x \in \mathbb{R}, 1-x^2 > m$.
		该命题为真,意味着存在一个实数 $x$,使得 $1-x^2$ 的值大于 $m$.
		这等价于说,$m$ 小于函数 $f(x) = 1-x^2$ 的最大值.
		函数 $f(x)=1-x^2$ 是一个开口向下的二次函数,其最大值为 $f(x)_{\max} = 1$ (当 $x=0$ 时取得).
		因此,实数 $m$ 的取值范围必须是 $m < 1$,即 $m \in (-\infty, 1)$.
	\end{solution}
	
	\subsubsection*{第 6 题}
	\begin{solution}
		\textbf{【答案】B}
		\paragraph{【解析】}
		“$p$ 是 $q$ 的充分不必要条件” 这句话包含两层信息:(1) $p \Rightarrow q$ (充分性);(2) $q \not\Rightarrow p$ (不必要性).必须同时检验这两个条件.
		设 $p: x=2$,$q: m^2x^2 - (m+3)x + 4 = 0$.
		\textbf{充分性 ($p \Rightarrow q$):} 若 $x=2$,则方程必须成立.代入 $x=2$ 得:
		$4m^2 - 2(m+3) + 4 = 0 \implies 2m^2 - m - 1 = 0 \implies (2m+1)(m-1) = 0$.
		解得 $m=1$ 或 $m=-\frac{1}{2}$.
		\textbf{不必要性 ($q \not\Rightarrow p$):} 方程的解集 $S$ 必须包含 $2$,但不能只有 $2$ 这一个解,即 $S \neq \{2\}$.
		当 $m=1$ 时,方程为 $x^2 - 4x + 4 = 0$, 解集 $S=\{2\}$. 此时 $q \Rightarrow p$ 成立,不满足“不必要”条件.故 $m=1$ 舍去.
		当 $m=-\frac{1}{2}$ 时,方程为 $\frac{1}{4}x^2 - \frac{5}{2}x + 4 = 0$, 即 $x^2 - 10x + 16 = 0$. 解集 $S=\{2, 8\}$. 此时 $x \in S$ 不能推出 $x=2$,满足“不必要”条件.
		综上,唯一满足条件的实数 $m$ 的值为 $-\frac{1}{2}$.
	\end{solution}
	
	\subsubsection*{第 7 题}
	\begin{solution}
		\textbf{【答案】D}
		\paragraph{【解析】}
		条件 $∁_U A = B$ 是解题的钥匙,它等价于 $A \cup B = U$ 且 $A \cap B = \emptyset$.利用这两个等价条件,结合集合元素的互异性,可以建立关于 $a,b$ 的约束关系.
		由 $∁_U A = B$,可知 $A$ 和 $B$ 的元素合起来恰好是全集 $U$,且它们之间没有公共元素.
		$A=\{1,a,b\}, B=\{4, a-b\}, U=\{1,2,3,4,5\}$.
		$A \cup B = \{1, a, b, 4, a-b\}$ 必须等于 $\{1,2,3,4,5\}$.
		比较两集合,可知 $\{a, b, a-b\}$ 必须是由 $\{2,3,5\}$ 这三个数构成,且满足互异性.
		我们只需在 $\{2,3,5\}$ 中为 $a, b, a-b$ 找到合适的配对.
		\textbf{情况一:} 设 $a=5, b=2$. 则 $a-b = 3$. 此时 $\{a,b,a-b\} = \{5,2,3\}$, 满足构成.
		检验:$A=\{1,5,2\}, B=\{4,3\}$, $∁_U A = \{3,4\}=B$. 相符.故 $(a,b)=(5,2)$ 是一组解.
		\textbf{情况二:} 设 $a=5, b=3$. 则 $a-b = 2$. 此时 $\{a,b,a-b\} = \{5,3,2\}$, 满足构成.
		检验:$A=\{1,5,3\}, B=\{4,2\}$, $∁_U A = \{2,4\}=B$. 相符.故 $(a,b)=(5,3)$ 是另一组解.
		其他情况(如 $a=3,b=2$)均不满足 $\{a,b,a-b\}=\{2,3,5\}$.
		所以 $a,b$ 的值为 5,2 或 5,3.
	\end{solution}
	
	\subsection*{C组}
	
	\subsubsection*{第 8 题}
	\begin{solution}
		\textbf{【答案】B}
		\paragraph{【解析】}
		本题首先要理解新定义,其次要运用集合元素的互异性,将“元素个数为3”转化为代数方程,解出参数 $n$ 的所有可能值,最后计算子集个数.
		\textbf{构造元素}
		$A=\{n, -1\}, B=\{\sqrt{2}, 1\}$.
		$A \otimes B = \{\sqrt{n^2 + (\sqrt{2})^2}, \sqrt{n^2 + 1^2}, \sqrt{(-1)^2 + (\sqrt{2})^2}, \sqrt{(-1)^2 + 1^2}\}$.
		$A \otimes B = \{\sqrt{n^2+2}, \sqrt{n^2+1}, \sqrt{3}, \sqrt{2}\}$.
		\textbf{应用元素个数条件}
		已知 $|A \otimes B| = 3$,而我们计算出了4个表达式.这意味着必有两个表达式的值相等.
		注意到 $\sqrt{n^2+2} > \sqrt{n^2+1}$ 且 $\sqrt{3} > \sqrt{2}$,所以相等的只可能是一个含 $n$ 的项与一个常数项.
		\textbf{情况 1:} $\sqrt{n^2+1} = \sqrt{2} \implies n^2=1 \implies n=\pm 1$. 此时 $\sqrt{n^2+2}=\sqrt{3}$, 集合为 $\{\sqrt{2}, \sqrt{3}\}$,2个元素,不合题意.
		\textbf{情况 2:} $\sqrt{n^2+1} = \sqrt{3} \implies n^2=2 \implies n=\pm \sqrt{2}$. 此时 $\sqrt{n^2+2}=2$, 集合为 $\{2, \sqrt{2}, \sqrt{3}\}$,3个元素,符合题意.
		\textbf{情况 3:} $\sqrt{n^2+2} = \sqrt{2} \implies n^2=0 \implies n=0$. 此时 $\sqrt{n^2+1}=1$, 集合为 $\{1, \sqrt{2}, \sqrt{3}\}$,3个元素,符合题意.
		\textbf{确定 $n$ 的取值集合并计算子集}
		由上可知,实数 $n$ 的所有取值组成的集合为 $S_n = \{0, \sqrt{2}, -\sqrt{2}\}$.
		该集合有 $k=3$ 个元素.
		非空真子集的个数为 $2^k - 2 = 2^3 - 2 = 6$.
	\end{solution}
	
	\subsection*{专题训练参考答案与解析}
	
	\subsubsection*{第 1 题}
	\begin{solution}
		\textbf{【答案】} $\{-3, 3\}$
		\paragraph{【解析】}
		本题考查有限集合的补集运算,可直接计算,也可利用德摩根定律检验.
		
		\textbf{方法一:直接法}
		首先确定全集 $U$ 和并集 $A \cup B$.
		$U = \{-3, -2, -1, 0, 1, 2, 3\}$.
		$A \cup B = \{-2, 1, 2\} \cup \{-1, 0, 1\} = \{-2, -1, 0, 1, 2\}$.
		$\complement_U(A \cup B)$ 即为在 $U$ 中但不在 $A \cup B$ 中的元素组成的集合.
		$\complement_U(A \cup B) = \{-3, 3\}$.
		
		\textbf{方法二:德摩根定律法}
		根据德摩根定律,$\complement_U(A \cup B) = (\complement_U A) \cap (\complement_U B)$.
		$\complement_U A = \{-3, -1, 0, 3\}$.
		$\complement_U B = \{-3, -2, 2, 3\}$.
		两个补集的交集为 $(\complement_U A) \cap (\complement_U B) = \{-3, 3\}$.
	\end{solution}
	
	\subsubsection*{第 2 题}
	\begin{solution}
		本题是容斥原理的直接应用.设参加数学兴趣小组的学生集合为 $M$,参加物理兴趣小组的学生集合为 $P$.
		根据题意,我们有:
		$|M| = 25$, $|P| = 22$, $|M \cap P| = 10$.
		(1) \textbf{求至少参加一个兴趣小组的人数},即求 $|M \cup P|$.
		根据容斥原理公式:
		$|M \cup P| = |M| + |P| - |M \cap P|$
		$= 25 + 22 - 10$
		$= 37$ (人).
		
		(2) \textbf{求两个小组都没有参加的人数}.
		全班人数构成全集 $U$,则所求人数为 $|\complement_U(M \cup P)|$.
		$|\complement_U(M \cup P)| = |U| - |M \cup P|$
		$= 50 - 37$
		$= 13$ (人).
	\end{solution}
	
	\subsubsection*{第 3 题}
	\begin{solution}
		本题将德摩根定律与解不等式结合,是典型的数形结合问题.
		(1) \textbf{求集合 B}
		解不等式 $x^2-3x-4<0$.
		因式分解得 $(x-4)(x+1)<0$.
		解得 $-1 < x < 4$.
		所以,集合 $B=\{x \mid -1 < x < 4\} = (-1, 4)$.
		
		(2) \textbf{求 $\complement_U(A \cap B)$}
		直接求 $A \cap B$ 较为复杂,我们应用德摩根定律将其转化为补集的并集运算.
		根据德摩根定律: $\complement_U(A \cap B) = (\complement_U A) \cup (\complement_U B)$.
		首先分别求 $\complement_U A$ 和 $\complement_U B$:
		由 $A=\{x|x \le -1 \text{ 或 } x \ge 4\}$, 得 $\complement_U A = \{x \mid -1 < x < 4\} = (-1, 4)$.
		由 $B=\{x|-1 < x < 4\}$, 得 $\complement_U B = \{x \mid x \le -1 \text{ 或 } x \ge 4\} = (-\infty, -1] \cup [4, \infty)$.
		接着求它们的并集:
		$(\complement_U A) \cup (\complement_U B) = (-1, 4) \cup ((-\infty, -1] \cup [4, \infty))$.
		在数轴上,这表示 $(-1, 4)$ 区间与该区间之外的所有点合并,结果为整个实数集.
		因此,$\complement_U(A \cap B) = \mathbb{R}$.
	\end{solution}
	
	\subsubsection*{第 4 题}
	\begin{solution}
		本题是容斥原理的逆向和综合应用.设数学及格的集合为 $M$,物理及格的集合为 $P$.
		已知 $|U|=50, |M|=40, |P|=31$.
		“两科都不及格”的人数为 4,即 $|\complement_U(M \cup P)|=4$.
		
		(1) \textbf{求两科都及格的人数},即求 $|M \cap P|$.
		首先,计算至少有一科及格的人数 $|M \cup P|$.
		$|M \cup P| = |U| - |\complement_U(M \cup P)| = 50 - 4 = 46$.
		根据容斥原理 $|M \cup P| = |M| + |P| - |M \cap P|$, 我们有:
		$46 = 40 + 31 - |M \cap P|$
		$|M \cap P| = 71 - 46 = 25$ (人).
		
		(2) \textbf{求恰好只有一科及格的人数}.
		这部分人由“数学及格但物理不及格”和“物理及格但数学不及格”两部分构成.可以从至少一科及格的总人数中,减去两科都及格的人数.
		恰好一科及格的人数 = $|M \cup P| - |M \cap P|$
		$= 46 - 25$
		$= 21$ (人).
	\end{solution}
	
	\subsubsection*{第 5 题}
	\begin{solution}
		本题是三集合容斥原理的经典应用,关键在于理解公式以及Venn图中各区域的含义.
		设关注 A, B, C 话题的人构成的集合分别为 $A, B, C$.
		$|A|=45, |B|=55, |C|=60$.
		$|A \cap B|=25, |A \cap C|=20, |B \cap C|=30$.
		$|A \cap B \cap C|=10$.
		
		(1) \textbf{求至少关注一个话题的人数},即求 $|A \cup B \cup C|$.
		套用三集合容斥原理公式:
		$|A \cup B \cup C| = (|A|+|B|+|C|) - (|A \cap B|+|A \cap C|+|B \cap C|) + |A \cap B \cap C|$
		$= (45+55+60) - (25+20+30) + 10$
		$= 160 - 75 + 10$
		$= 95$ (人).
		
		(2) \textbf{求恰好只关注一个话题的人数}.
		我们可以分别计算只关注A、只关注B、只关注C的人数,然后相加.Venn图有助于我们思考这个减法过程.
		只关注A的人数 = $|A| - |A \cap B| - |A \cap C| + |A \cap B \cap C|$
		$= 45 - 25 - 20 + 10 = 10$ (人).
		解释:从A中减去AB交和AC交时,三者公共部分被减了两次,故需加回一次.
		只关注B的人数 = $|B| - |A \cap B| - |B \cap C| + |A \cap B \cap C|$
		$= 55 - 25 - 30 + 10 = 10$ (人).
		只关注C的人数 = $|C| - |A \cap C| - |B \cap C| + |A \cap B \cap C|$
		$= 60 - 20 - 30 + 10 = 20$ (人).
		所以,恰好只关注一个话题的总人数 = $10 + 10 + 20 = 40$ (人).
	\end{solution}
	
	\section{解三角形}
	\subsection{取值范围}

	\subsubsection{习题1}
	\begin{solution}
		本题的核心数学对象是 $\triangle ABC$,回顾我们所学的知识,一个三角形的\textbf{形状}由其三个内角 $A, B, C$ 决定.由于内角和为 $\pi$,即 $A+B+C=\pi$,,是为其一之约束,因此,一个三角形的形状本质上只有 \textbf{2 个自由度}.我们可以选择任意两个角(例如 $A$ 和 $B$)作为独立的参数,第三个角 $C$ 便随之确定.
		
		题目给出的条件 $\frac{\cos A}{1+\sin A} = \frac{\sin 2B}{1+\cos 2B}$ 是一个满足式条件,而我们的首要任务,就是将这个不咋好看的等式,翻译成一个关于角 $A, B$ 的更简洁的关系式.
		
		因此,我们化简下原式,在等式左边,利用半角公:
		\[
		\frac{\cos A}{1+\sin A} = \frac{\sin(\frac{\pi}{2}-A)}{1+\cos(\frac{\pi}{2}-A)} = \frac{2\sin(\frac{\pi}{4}-\frac{A}{2})\cos(\frac{\pi}{4}-\frac{A}{2})}{2\cos^2(\frac{\pi}{4}-\frac{A}{2})} = \tan\left(\frac{\pi}{4}-\frac{A}{2}\right)
		\]
		等式右边,利用二倍角公式:
		\[
		\frac{\sin 2B}{1+\cos 2B} = \frac{2\sin B \cos B}{1+(2\cos^2 B-1)} = \frac{2\sin B \cos B}{2\cos^2 B} = \tan B
		\]
		因此:
		\[
		\tan\left(\frac{\pi}{4}-\frac{A}{2}\right) = \tan B
		\]
		由于 $A, B \in (0, \pi)$, 可知 $\frac{\pi}{4}-\frac{A}{2} \in (-\frac{\pi}{4}, \frac{\pi}{4})$.在各自的定义域内,正切函数是单调的,故:
		\[
		\frac{\pi}{4}-\frac{A}{2} = B \quad \implies \quad A+2B = \frac{\pi}{2}
		\]
		这个简洁的关系,就是原复杂三角等式背后隐藏的约束.它将三角形形状的 2 个自由度削减为了 \textbf{1 个自由度}.接着,我们只需要确定一个角,整个三角形的形状就完全确定了.
		
		\textbf{对于第 (1) 问:}
		题目给出了构造式条件 $C = \frac{2\pi}{3}$,这提供了消除最后一个自由度的信息.我们现在有两个关于 $A, B, C$ 的线性约束:
		\begin{align*}
			A+B+C &= \pi \quad  \\
			A+2B \qquad &= \frac{\pi}{2} \quad 
		\end{align*}
		将 $C=\frac{2\pi}{3}$ 代入第一个式子,得到 $A+B = \frac{\pi}{3}$.联立方程组:
		\[
		\begin{cases}
			A+2B = \frac{\pi}{2} \\
			A+B = \frac{\pi}{3}
		\end{cases}
		\]
		解得 $B = \frac{\pi}{6}$.至此,三角形的所有角都被唯一确定,问题解决.
		
		\textbf{对于第 (2) 问:}
		此问没有给出额外条件,因此我们需要在由 $A+2B=\frac{\pi}{2}$ 所确定的\textbf{所有可能}的三角形形状中,寻找目标表达式的最小值.这正是用自由度参数表达并求解思想的用武之地.
		
		我们选择角 $B$ 作为描述三角形形状的唯一自由参数.其他角可以用 $B$ 表示:
		\begin{align*}
			A &= \frac{\pi}{2}-2B \\
			C &= \pi - (A+B) = \pi - \left(\left(\frac{\pi}{2}-2B\right)+B\right) = \frac{\pi}{2}+B
		\end{align*}
		为了使 $\triangle ABC$ 成立,所有内角必须为正:
		\[
		A > 0 \implies \frac{\pi}{2}-2B > 0 \implies B < \frac{\pi}{4}
		\]
		\[
		B > 0 \quad \text{且} \quad C = \frac{\pi}{2}+B > 0 \quad (\text{当 } B>0 \text{ 时恒成立})
		\]
		综上,自由参数 $B$ 的取值范围是 $(0, \frac{\pi}{4})$.
		
		目标函数是 $\frac{a^2+b^2}{c^2}$.利用正弦定理,将其从边长语言翻译为角度语言:
		\[
		\frac{a^2+b^2}{c^2} = \frac{\sin^2 A + \sin^2 B}{\sin^2 C}
		\]
		将所有角用参数 $B$ 代入:
		\[
		f(B) = \frac{\sin^2(\frac{\pi}{2}-2B) + \sin^2 B}{\sin^2(\frac{\pi}{2}+B)} = \frac{\cos^2(2B) + \sin^2 B}{\cos^2 B}
		\]
		问题转化为:求函数 $f(B)$ 在区间 $B \in (0, \frac{\pi}{4})$ 上的最小值.
		
		为了简化计算,我们进行变量代换.这是一个典型的换元法应用,其本质是将问题从三角函数空间映射到更易于处理的代数空间.
		令 $x = \sin^2 B$.
		由于 $B \in (0, \frac{\pi}{4})$, 新变量 $x$ 的取值范围是 $(0, \sin^2(\frac{\pi}{4})) = (0, \frac{1}{2})$.
		我们将 $f(B)$ 的表达式完全用 $x$ 来表示:
		\[
		\cos^2 B = 1-\sin^2 B = 1-x \quad \text{且} \quad \cos(2B) = 1-2\sin^2 B = 1-2x
		\]
		代入得到关于 $x$ 的函数 $g(x)$:
		\[
		g(x) = \frac{(1-2x)^2+x}{1-x} = \frac{4x^2-3x+1}{1-x}
		\]
		为求 $g(x)$ 在区间 $(0, \frac{1}{2})$ 上的最小值,我们对其求导:
		\[
		g'(x) = \frac{(8x-3)(1-x) - (4x^2-3x+1)(-1)}{(1-x)^2} = \frac{-4x^2+8x-2}{(1-x)^2}
		\]
		令 $g'(x)=0$, 即 $2x^2-4x+1 = 0$.解得 $x = 1 \pm \frac{\sqrt{2}}{2}$.
		考虑到定义域 $x \in (0, \frac{1}{2})$,我们取驻点 $x_0 = 1 - \frac{\sqrt{2}}{2}$.
		分析可知,当 $x \in (0, x_0)$ 时,$g'(x)<0$;当 $x \in (x_0, \frac{1}{2})$ 时,$g'(x)>0$.因此 $g(x)$ 在 $x=x_0$ 处取得最小值.
		将 $x_0 = 1 - \frac{\sqrt{2}}{2}$ 代入 $g(x)$.为简化计算,先对 $g(x)$ 进行代数变形:
		\[
		g(x) = \frac{-4x(1-x)+(x-1)+2}{1-x} = -4x-1+\frac{2}{1-x}
		\]
		最小值为:
		\begin{align*}
			g\left(1 - \frac{\sqrt{2}}{2}\right) &= -4\left(1-\frac{\sqrt{2}}{2}\right) - 1 + \frac{2}{1-\left(1-\frac{\sqrt{2}}{2}\right)} \\
			&= -4 + 2\sqrt{2} - 1 + \frac{2}{\frac{\sqrt{2}}{2}} \\
			&= -5 + 2\sqrt{2} + 2\sqrt{2} \\
			&= 4\sqrt{2}-5
		\end{align*}\hfill\qedsymbol
	\end{solution}
	
	\subsubsection{习题2}
	\begin{solution}
		题目给出的核心约束是关于边和角的满足式条件”$a=b+2b\cos C$.为了得到我们想要的关系,我们先利用正弦定理翻译起手,目的是将边长关系翻译为角度关系.
		
		由正弦定理,设 $a=k\sin A, b=k\sin B, c=k\sin C$.代入原式得:
		\[
		k\sin A = k\sin B + 2k\sin B \cos C
		\]
		由于 $k \neq 0$, 两边同除以 $k$ 得:
		\[
		\sin A = \sin B + 2\sin B \cos C
		\]
		在 $\triangle ABC$ 中, $A = \pi-(B+C)$, 故 $\sin A = \sin(\pi-(B+C)) = \sin(B+C)$.代入上式:
		\[
		\sin(B+C) = \sin B + 2\sin B \cos C
		\]
		展开左侧:
		\[
		\sin B \cos C + \cos B \sin C = \sin B + 2\sin B \cos C
		\]
		移项整理得:
		\[
		\cos B \sin C - \sin B \cos C = \sin B
		\]
		左侧恰为差角公式,即 $\sin(C-B) = \sin B$.
		
		由于 $B,C$ 均为三角形内角,故 $B \in (0,\pi), C \in (0,\pi)$.同时 $B+C \in (0,\pi)$, 可得 $C-B \in (-\pi, \pi)$.
		由 $\sin(C-B)=\sin B$ 可得两种可能:
		\begin{itemize}
			\item $C-B = B$, 即 $C=2B$.
			\item $C-B = \pi - B$, 即 $C=\pi$.矛盾.
		\end{itemize}
		因此, $C=2B$.至此,第 (1) 问得证.
		
		第 (1) 问的结论 $C=2B$ 极大地简化了问题,它将三角形形状的自由度从 2 降至 1.我们可以选择角 $B$ 作为唯一的自由参数来描述所有满足条件的三角形.
		
		首先,我们将所有角用参数 $B$ 表示,也即所谓的选定变量,或者说是归一思想:
		\[
		C=2B, \quad A = \pi - (B+C) = \pi - 3B
		\]
		为保证 $A,B,C$ 构成一个三角形,所有内角必须为正:
		\begin{align*}
			B &> 0 \\
			C = 2B &> 0 \quad (\text{当 } B>0 \text{ 时恒成立}) \\
			A = \pi - 3B &> 0 \implies 3B < \pi \implies B < \frac{\pi}{3}
		\end{align*}
		综上,自由参数 $B$ 的取值范围是 $(0, \frac{\pi}{3})$.
		
	目标函数为 $\frac{a+c}{b}$.我们再次利用正弦定理,将其翻译为角度的表达式:
		\[
		\frac{a+c}{b} = \frac{k\sin A + k\sin C}{k\sin B} = \frac{\sin A + \sin C}{\sin B}
		\]
		将所有角用参数 $B$ 代入:
		\[
		f(B) = \frac{\sin(\pi-3B)+\sin(2B)}{\sin B}
		\]
		由于 $B \in (0, \frac{\pi}{3})$, $\sin B \neq 0$,我们可以进行化简:
		\[
		f(B) = \frac{\sin(3B)+\sin(2B)}{\sin B}
		\]
		利用和差化积公式或三倍角、二倍角公式展开.这里使用后者更为直接:
		\begin{align*}
			f(B) &= \frac{(3\sin B - 4\sin^3 B) + (2\sin B \cos B)}{\sin B} \\
			&= 3 - 4\sin^2 B + 2\cos B
		\end{align*}
		
		为了求 $f(B)$ 的范围,我们将其转化为关于单一三角函数变量的代数式.
		令 $t = \cos B$.首先确定新变量 $t$ 的范围.
		因为 $B \in (0, \frac{\pi}{3})$ 且 $y=\cos x$ 在此区间上单调递减,所以:
		\[
		t = \cos B \in \left(\cos\frac{\pi}{3}, \cos 0\right) = \left(\frac{1}{2}, 1\right)
		\]
		将原表达式中的 $\sin^2 B$ 替换为 $1-\cos^2 B = 1-t^2$:
		\begin{align*}
			g(t) &= 3 - 4(1-t^2) + 2t \\
			&= 3 - 4 + 4t^2 + 2t \\
			&= 4t^2 + 2t - 1
		\end{align*}
		问题最终转化为:求二次函数 $g(t) = 4t^2+2t-1$ 在开区间 $t \in (\frac{1}{2}, 1)$ 上的值域.
		
		这是一个开口向上的抛物线,对称轴为 $t = -\frac{2}{2 \times 4} = -\frac{1}{4}$.
		由于区间 $(\frac{1}{2}, 1)$ 完全位于对称轴的右侧,所以函数 $g(t)$ 在此区间上是严格单调递增的.
		
		因此,值域的边界由区间端点决定:
		\begin{itemize}
			\item 当 $t \to \frac{1}{2}^+$ 时, $g(t) \to 4(\frac{1}{2})^2 + 2(\frac{1}{2}) - 1 = 1+1-1=1$.
			\item 当 $t \to 1^-$ 时, $g(t) \to 4(1)^2 + 2(1) - 1 = 4+2-1=5$.
		\end{itemize}
		由于 $t$ 的取值范围是开区间,所以函数值的范围也是开区间.
		
		最终,$\frac{a+c}{b}$ 的取值范围是 $(1, 5)$.\hfill\qedsymbol
	\end{solution}
	
	\subsubsection{习题3}
	\begin{solution}
		本题通过一个面积与边长的关系式来约束三角形的形状.我们的策略是利用面积公式、余弦定理来翻译这个关系式,从而确定一个内角.
		
		\textbf{1. 求角$B$.}
		一方面,三角形的面积公式为:
		\[
		S = \frac{1}{2}ac\sin B
		\]
		另一方面,题目给出的条件 $S = \frac{\sqrt{3}}{4}(a^2+c^2-b^2)$ 中,括号内的项 $a^2+c^2-b^2$ 与余弦定理结构相似.
		由余弦定理,$b^2 = a^2+c^2 - 2ac\cos B$, 可得:
		\[
		a^2+c^2-b^2 = 2ac\cos B
		\]
		将此式代入题目给定的条件中:
		\[
		S = \frac{\sqrt{3}}{4}(2ac\cos B) = \frac{\sqrt{3}}{2}ac\cos B
		\]
		于是得到了关于面积 $S$ 的两个等价表达式,令它们相等:
		\[
		\frac{1}{2}ac\sin B = \frac{\sqrt{3}}{2}ac\cos B
		\]
		由于在三角形中 $a, c$ 均为正数,可约去 $\frac{1}{2}ac$:
		\[
		\sin B = \sqrt{3}\cos B
		\]
		因为 $B \in (0, \pi)$, $\cos B=0$ 时 $\sin B \neq 0$, 故 $\cos B \neq 0$.两边同除以 $\cos B$ 得:
		\[
		\tan B = \sqrt{3}
		\]
		考虑到 $B$ 是三角形内角,解得 $B = \frac{\pi}{3}$.
		
		\textbf{2. 求解 $\triangle AOC$ 周长的取值范围}
		
		第 (1) 问确定了角 $B$ 的大小,多了一个条件,方便我们弄出第二问,来看第二问,其目标对象是 $\triangle AOC$ 的周长,我们的思路是:先将周长表达为一个函数的形式,并确定其定义域,进而求出值域.
		
		点 $O$ 是 $\triangle ABC$ 的内心,即 $AO, CO$ 分别是 $\angle A$ 和 $\angle C$ 的角平分线.因此,在 $\triangle AOC$ 中:
		\[
		\angle OAC = \frac{A}{2}, \quad \angle OCA = \frac{C}{2}
		\]
		其第三个角为:
		\[
		\angle AOC = \pi - \left(\frac{A}{2} + \frac{C}{2}\right) = \pi - \frac{A+C}{2}
		\]
		由于我们已经求出 $B=\frac{\pi}{3}$, 所以 $A+C = \pi - B = \frac{2\pi}{3}$.代入上式,我们得到一个重要的定值:
		\[
		\angle AOC = \pi - \frac{2\pi/3}{2} = \pi - \frac{\pi}{3} = \frac{2\pi}{3}
		\]
		$\triangle AOC$ 的周长 $P = AO + CO + AC$.其中 $AC = b = 2\sqrt{3}$ 是已知定值.
		
		我们选择角 $A$ 作为唯一的自由参数.在 $\triangle AOC$ 中,应用正弦定理:
		\[
		\frac{AO}{\sin(\angle OCA)} = \frac{CO}{\sin(\angle OAC)} = \frac{AC}{\sin(\angle AOC)}
		\]
		\[
		\frac{AO}{\sin(C/2)} = \frac{CO}{\sin(A/2)} = \frac{2\sqrt{3}}{\sin(2\pi/3)} = \frac{2\sqrt{3}}{\sqrt{3}/2} = 4
		\]
		由此解出 $AO = 4\sin(C/2)$ 和 $CO = 4\sin(A/2)$.
		周长表达式为:
		\[
		P(A,C) = 4\sin(A/2) + 4\sin(C/2) + 2\sqrt{3}
		\]
		将 $C = \frac{2\pi}{3}-A$ 代入,得到关于单参数 $A$ 的函数:
		\[
		P(A) = 4\left[\sin\left(\frac{A}{2}\right) + \sin\left(\frac{2\pi/3-A}{2}\right)\right] + 2\sqrt{3} = 4\left[\sin\left(\frac{A}{2}\right) + \sin\left(\frac{\pi}{3}-\frac{A}{2}\right)\right] + 2\sqrt{3}
		\]
		
		首先确定参数 $A$ 的范围.在 $\triangle ABC$ 中,各角需为正:
		\[
		A>0, \quad C = \frac{2\pi}{3}-A > 0 \implies A < \frac{2\pi}{3}
		\]
		故 $A \in (0, \frac{2\pi}{3})$.
		
		我们对括号内的三角函数部分进行化简,令其为 $g(A) = \sin(\frac{A}{2}) + \sin(\frac{\pi}{3}-\frac{A}{2})$.
		利用和差化积公式:
		\[
		g(A) = 2\sin\left(\frac{A/2 + \pi/3-A/2}{2}\right)\cos\left(\frac{A/2 - (\pi/3-A/2)}{2}\right) = 2\sin\left(\frac{\pi}{6}\right)\cos\left(\frac{A-\pi/3}{2}\right)
		\]
		由于 $2\sin(\frac{\pi}{6}) = 2 \cdot \frac{1}{2} = 1$, 所以 $g(A) = \cos\left(\frac{A}{2}-\frac{\pi}{6}\right)$.
		
		现在求 $g(A)$ 的范围.因为 $A \in (0, \frac{2\pi}{3})$:
		\[
		\frac{A}{2} \in \left(0, \frac{\pi}{3}\right) \implies \frac{A}{2}-\frac{\pi}{6} \in \left(-\frac{\pi}{6}, \frac{\pi}{6}\right)
		\]
		余弦函数 $y=\cos x$ 在区间 $(-\frac{\pi}{6}, \frac{\pi}{6})$ 上是偶函数,且在该区间上从 $\cos(-\pi/6)=\frac{\sqrt{3}}{2}$ 减到 $\cos(0)=1$ 再增回 $\cos(\pi/6)=\frac{\sqrt{3}}{2}$.其值域为 $(\cos(\frac{\pi}{6}), \cos(0)] = (\frac{\sqrt{3}}{2}, 1]$.
		所以 $g(A) \in (\frac{\sqrt{3}}{2}, 1]$.
		
		最终,周长 $P(A) = 4g(A) + 2\sqrt{3}$ 的取值范围是:
		\[
		\left(4 \cdot \frac{\sqrt{3}}{2} + 2\sqrt{3}, \quad 4 \cdot 1 + 2\sqrt{3}\right] = \left(2\sqrt{3} + 2\sqrt{3}, \quad 4 + 2\sqrt{3}\right] = \left(4\sqrt{3}, 4+2\sqrt{3}\right]
		\]
		故 $\triangle AOC$ 周长的取值范围是 $(4\sqrt{3}, 4+2\sqrt{3}]$.\hfill\qedsymbol
	\end{solution}
	
	\subsubsection{习题4}
	\begin{solution}
		\textbf{1. 求角$B$.}
		
		题目给出的条件是一个混合了边与角的复杂比例式.我们的首要策略是利用正弦定理,将所有三角函数项统一替换为边长,从而将问题转化为纯粹的代数关系式,以揭示约束条件.
		
		由正弦定理,$\sin A = \frac{a}{2R}, \sin B = \frac{b}{2R}, \sin C = \frac{c}{2R}$.代入原式:
		\[
		\frac{\frac{a}{2R} - \frac{b}{2R}}{\sqrt{3}a - c} = \frac{\frac{c}{2R}}{a+b}
		\]
		消去公因式 $\frac{1}{2R}$,我们得到:
		\[
		\frac{a-b}{\sqrt{3}a - c} = \frac{c}{a+b}
		\]
		通过交叉相乘,将比例式转化为等式:
		\[
		(a-b)(a+b) = c(\sqrt{3}a - c)
		\]
		展开得:
		\[
		a^2 - b^2 = \sqrt{3}ac - c^2
		\]
		移项,整理成余弦定理的形式:
		\[
		a^2 + c^2 - b^2 = \sqrt{3}ac
		\]
		根据余弦定理,我们知道 $a^2 + c^2 - b^2 = 2ac\cos B$.于是:
		\[
		2ac\cos B = \sqrt{3}ac
		\]
		由于 $a,c$ 是三角形的边长,必为正数,可约去 $2ac$:
		\[
		\cos B = \frac{\sqrt{3}}{2}
		\]
		因为 $B$ 是三角形内角,所以 $B = \frac{\pi}{6}$.这个值满足锐角三角形中 $B \in (0, \frac{\pi}{2})$ 的要求.
		
		\textbf{2. 求解周长的取值范围}
		
		第 (1) 问固定了角 $B=\frac{\pi}{6}$,第 (2) 问又给定了边 $a=2$.此时,三角形的自由度仅剩一个.我们可以选择一个角(例如角 $A$)作为自由参数,来表达周长并确定其范围.
		
	我们选择角 $A$ 为参数.$\triangle ABC$ 是锐角三角形,因此所有角都必须在 $(0, \frac{\pi}{2})$ 内.
		\begin{itemize}
			\item $A \in (0, \frac{\pi}{2})$
			\item $B = \frac{\pi}{6} \in (0, \frac{\pi}{2})$ 
			\item $C = \pi - (A+B) = \pi - (A+\frac{\pi}{6}) = \frac{5\pi}{6} - A \in (0, \frac{\pi}{2})$
		\end{itemize}
		从 $C$ 的范围我们可以得到对 $A$ 的两个约束:
		\begin{itemize}
			\item $\frac{5\pi}{6} - A > 0 \implies A < \frac{5\pi}{6}$.此条件被 $A < \pi/2$ 包含.
			\item $\frac{5\pi}{6} - A < \frac{\pi}{2} \implies \frac{5\pi}{6} - \frac{3\pi}{6} < A \implies A > \frac{2\pi}{6} = \frac{\pi}{3}$.
		\end{itemize}
		综上,参数 $A$ 的取值范围是 $(\frac{\pi}{3}, \frac{\pi}{2})$.
		
		确定完后,我们看看周长 $P = a+b+c = 2+b+c$,利用正弦定理将 $b,c$ 用参数 $A$ 表示.
		\[
		\frac{a}{\sin A} = \frac{b}{\sin B} = \frac{c}{\sin C} \implies \frac{2}{\sin A} = \frac{b}{\sin(\pi/6)} = \frac{c}{\sin(5\pi/6 - A)}
		\]
		解得:
		\[
		b = \frac{2\sin(\pi/6)}{\sin A} = \frac{1}{\sin A}
		\]
		\[
		c = \frac{2\sin(5\pi/6 - A)}{\sin A}
		\]
		所以,周长 $P$ 是关于 $A$ 的函数:
		\begin{align*}
			P(A) &= 2 + \frac{1}{\sin A} + \frac{2\sin(5\pi/6 - A)}{\sin A} \\
			&= 2 + \frac{1 + 2\left(\sin\frac{5\pi}{6}\cos A - \cos\frac{5\pi}{6}\sin A\right)}{\sin A} \\
			&= 2 + \frac{1 + 2\left(\frac{1}{2}\cos A + \frac{\sqrt{3}}{2}\sin A\right)}{\sin A} \\
			&= 2 + \frac{1 + \cos A + \sqrt{3}\sin A}{\sin A} \\
			&= 2 + \frac{1+\cos A}{\sin A} + \sqrt{3}
		\end{align*}
		
	化简,然后分析函数 $g(A) = \frac{1+\cos A}{\sin A}$.利用半角公式,简单期间,设$\cot A/2 = \frac{1}{\tan A/2}$,这里用到余切函数了,当然你换个元也是一样的效果.
		\[
		g(A) = \frac{2\cos^2(A/2)}{2\sin(A/2)\cos(A/2)} = \cot(A/2)
		\]
		因此,周长函数为 $P(A) = 2 + \sqrt{3} + \cot(A/2)$.
		
		注意道当 $A \in (\frac{\pi}{3}, \frac{\pi}{2})$ 时,
		\[
		A \in \left(\frac{\pi}{3}, \frac{\pi}{2}\right) \implies \frac{A}{2} \in \left(\frac{\pi}{6}, \frac{\pi}{4}\right)
		\]
		函数 $y = \cot x$ 在区间 $(\frac{\pi}{6}, \frac{\pi}{4})$ 上是单减.
		因此,函数的值域由区间端点决定:
		\begin{itemize}
			\item 当 $\frac{A}{2} \to \frac{\pi}{6}^+$ 时,$\cot(\frac{A}{2}) \to \cot(\frac{\pi}{6}) = \sqrt{3}$.
			\item 当 $\frac{A}{2} \to \frac{\pi}{4}^-$ 时,$\cot(\frac{A}{2}) \to \cot(\frac{\pi}{4}) = 1$.
		\end{itemize}
		所以,$\cot(A/2)$ 的取值范围是 $(1, \sqrt{3})$.
		
		综上,周长 $P(A) = 2+\sqrt{3} + \cot(A/2)$ 的取值范围是:
		\[
		\left(2+\sqrt{3} + 1, \quad 2+\sqrt{3} + \sqrt{3}\right) = \left(3+\sqrt{3}, 2+2\sqrt{3}\right)
		\]
		故 $\triangle ABC$ 周长的取值范围是 $(3+\sqrt{3}, 2+2\sqrt{3})$.\hfill\qedsymbol
	\end{solution}
	
	\subsection{面积问题}
	\subsubsection{习题5}
	\begin{solution}
		\textbf{1. 求 $\angle ADC$}
		
		本题要我们求 $\triangle ADC$.题目直接给出了这个三角形的三条边长:$AD=5, DC=3, AC=7$.当一个三角形的三边确定时,其形状和所有内角也随之完全确定.
		
		先应用余弦定理来求解 $\angle ADC$.在 $\triangle ADC$ 中:
		\[
		AC^2 = AD^2 + DC^2 - 2(AD)(DC)\cos(\angle ADC)
		\]
		代入已知数据:
		\[
		7^2 = 5^2 + 3^2 - 2(5)(3)\cos(\angle ADC)
		\]
		\[
		49 = 25 + 9 - 30\cos(\angle ADC)
		\]
		\[
		49 = 34 - 30\cos(\angle ADC)
		\]
		整理得到:
		\[
		15 = -30\cos(\angle ADC) \implies \cos(\angle ADC) = -\frac{1}{2}
		\]
		由于 $\angle ADC$ 是三角形的内角,其范围为 $(0, \pi)$,因此 $\angle ADC = \frac{2\pi}{3}$.
		
		\textbf{2. 求 $\triangle ABC$ 面积}
		
		为求 $\triangle ABC$ 的面积,我们需要确定更多的边或角.关键在于利用连接两个三角形的条件 $\angle DAC = \angle ABC$.
		
		令 $\angle ABC = B$, 则 $\angle DAC = B$.
		在 $\triangle ADC$ 中,我们已知三边和一角,可以利用正弦定理求出 $\sin B$:
		\[
		\frac{AC}{\sin(\angle ADC)} = \frac{DC}{\sin(\angle DAC)}
		\]
		\[
		\frac{7}{\sin(2\pi/3)} = \frac{3}{\sin B} \implies \sin B = \frac{3\sin(2\pi/3)}{7} = \frac{3(\sqrt{3}/2)}{7} = \frac{3\sqrt{3}}{14}
		\]
		
		现在我们的焦点转移到 $\triangle ABD$.
		我们知道 $\angle ADC = \frac{2\pi}{3}$, 故其补角 $\angle ADB = \pi - \frac{2\pi}{3} = \frac{\pi}{3}$.
		在 $\triangle ABD$ 中,我们已知:
		\begin{itemize}
			\item 边 $AD = 5$
			\item 角 $\angle ABD = B$
			\item 角 $\angle ADB = \frac{\pi}{3}$
		\end{itemize}
		第三个角为 $\angle BAD = \pi - B - \frac{\pi}{3} = \frac{2\pi}{3}-B$.
		应用正弦定理于 $\triangle ABD$:
		\[
		\frac{AB}{\sin(\angle ADB)} = \frac{BD}{\sin(\angle BAD)} = \frac{AD}{\sin(\angle ABD)}
		\]
		\[
		\frac{AB}{\sin(\pi/3)} = \frac{BD}{\sin(2\pi/3 - B)} = \frac{5}{\sin B}
		\]
		由此可解得边长 $BD$:
		\[
		BD = \frac{5\sin(2\pi/3 - B)}{\sin B}
		\]
		为计算此值,我们需要 $\cos B$.由于 $\triangle ABC$ 中 $D$ 在 $BC$ 边上,则 $C = \angle ACD$ 必为锐角(否则 $C+\angle ADC \ge \pi$),故 $\angle DAC = B$ 也必为锐角.因此 $\cos B > 0$.
		\[
		\cos B = \sqrt{1-\sin^2 B} = \sqrt{1 - \left(\frac{3\sqrt{3}}{14}\right)^2} = \sqrt{1-\frac{27}{196}} = \sqrt{\frac{169}{196}} = \frac{13}{14}
		\]
		展开 $\sin(2\pi/3 - B)$:
		\begin{align*}
			BD &= \frac{5(\sin(2\pi/3)\cos B - \cos(2\pi/3)\sin B)}{\sin B} \\
			&= \frac{5((\sqrt{3}/2)(13/14) - (-1/2)(3\sqrt{3}/14))}{3\sqrt{3}/14} \\
			&= \frac{5(13\sqrt{3}/28 + 3\sqrt{3}/28)}{3\sqrt{3}/14} = \frac{5(16\sqrt{3}/28)}{3\sqrt{3}/14} = \frac{5(4\sqrt{3}/7)}{3\sqrt{3}/14} = \frac{20\sqrt{3}}{7} \cdot \frac{14}{3\sqrt{3}} = \frac{40}{3}
		\end{align*}

		接下来,计算面积即可,$\triangle ABC$ 的面积等于 $\triangle ABD$ 与 $\triangle ADC$ 的面积之和.
		\[
		S_{\triangle ADC} = \frac{1}{2}AD \cdot DC \sin(\angle ADC) = \frac{1}{2}(5)(3)\sin\left(\frac{2\pi}{3}\right) = \frac{15}{2} \cdot \frac{\sqrt{3}}{2} = \frac{15\sqrt{3}}{4}
		\]
		\[
		S_{\triangle ABD} = \frac{1}{2}AD \cdot BD \sin(\angle ADB) = \frac{1}{2}(5)\left(\frac{40}{3}\right)\sin\left(\frac{\pi}{3}\right) = \frac{100}{3} \cdot \frac{\sqrt{3}}{2} = \frac{50\sqrt{3}}{3}
		\]
		总面积为:
		\[
		S_{\triangle ABC} = S_{\triangle ADC} + S_{\triangle ABD} = \frac{15\sqrt{3}}{4} + \frac{50\sqrt{3}}{3} = \sqrt{3}\left(\frac{45+200}{12}\right) = \frac{245\sqrt{3}}{12}
		\]
		故 $\triangle ABC$ 的面积为 $\frac{245\sqrt{3}}{12}$.\hfill\qedsymbol
	\end{solution}
\end{document}