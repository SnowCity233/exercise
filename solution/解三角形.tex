\section{解三角形}
\subsection{取值范围}

\subsubsection{习题1}
\begin{solution}
	本题的核心数学对象是 $\triangle ABC$,回顾我们所学的知识,一个三角形的\textbf{形状}由其三个内角 $A, B, C$ 决定.由于内角和为 $\pi$,即 $A+B+C=\pi$,,是为其一之约束,因此,一个三角形的形状本质上只有 \textbf{2 个自由度}.我们可以选择任意两个角(例如 $A$ 和 $B$)作为独立的参数,第三个角 $C$ 便随之确定.
	
	题目给出的条件 $\frac{\cos A}{1+\sin A} = \frac{\sin 2B}{1+\cos 2B}$ 是一个满足式条件,而我们的首要任务,就是将这个不咋好看的等式,翻译成一个关于角 $A, B$ 的更简洁的关系式.
	
	因此,我们化简下原式,在等式左边,利用半角公:
	\[
	\frac{\cos A}{1+\sin A} = \frac{\sin(\frac{\pi}{2}-A)}{1+\cos(\frac{\pi}{2}-A)} = \frac{2\sin(\frac{\pi}{4}-\frac{A}{2})\cos(\frac{\pi}{4}-\frac{A}{2})}{2\cos^2(\frac{\pi}{4}-\frac{A}{2})} = \tan\left(\frac{\pi}{4}-\frac{A}{2}\right)
	\]
	等式右边,利用二倍角公式:
	\[
	\frac{\sin 2B}{1+\cos 2B} = \frac{2\sin B \cos B}{1+(2\cos^2 B-1)} = \frac{2\sin B \cos B}{2\cos^2 B} = \tan B
	\]
	因此:
	\[
	\tan\left(\frac{\pi}{4}-\frac{A}{2}\right) = \tan B
	\]
	由于 $A, B \in (0, \pi)$, 可知 $\frac{\pi}{4}-\frac{A}{2} \in (-\frac{\pi}{4}, \frac{\pi}{4})$.在各自的定义域内,正切函数是单调的,故:
	\[
	\frac{\pi}{4}-\frac{A}{2} = B \quad \implies \quad A+2B = \frac{\pi}{2}
	\]
	这个简洁的关系,就是原复杂三角等式背后隐藏的约束.它将三角形形状的 2 个自由度削减为了 \textbf{1 个自由度}.接着,我们只需要确定一个角,整个三角形的形状就完全确定了.
	
	\textbf{对于第 (1) 问:}
	题目给出了构造式条件 $C = \frac{2\pi}{3}$,这提供了消除最后一个自由度的信息.我们现在有两个关于 $A, B, C$ 的线性约束:
	\begin{align*}
		A+B+C &= \pi \quad  \\
		A+2B \qquad &= \frac{\pi}{2} \quad 
	\end{align*}
	将 $C=\frac{2\pi}{3}$ 代入第一个式子,得到 $A+B = \frac{\pi}{3}$.联立方程组:
	\[
	\begin{cases}
		A+2B = \frac{\pi}{2} \\
		A+B = \frac{\pi}{3}
	\end{cases}
	\]
	解得 $B = \frac{\pi}{6}$.至此,三角形的所有角都被唯一确定,问题解决.
	
	\textbf{对于第 (2) 问:}
	此问没有给出额外条件,因此我们需要在由 $A+2B=\frac{\pi}{2}$ 所确定的\textbf{所有可能}的三角形形状中,寻找目标表达式的最小值.这正是用自由度参数表达并求解思想的用武之地.
	
	我们选择角 $B$ 作为描述三角形形状的唯一自由参数.其他角可以用 $B$ 表示:
	\begin{align*}
		A &= \frac{\pi}{2}-2B \\
		C &= \pi - (A+B) = \pi - \left(\left(\frac{\pi}{2}-2B\right)+B\right) = \frac{\pi}{2}+B
	\end{align*}
	为了使 $\triangle ABC$ 成立,所有内角必须为正:
	\[
	A > 0 \implies \frac{\pi}{2}-2B > 0 \implies B < \frac{\pi}{4}
	\]
	\[
	B > 0 \quad \text{且} \quad C = \frac{\pi}{2}+B > 0 \quad (\text{当 } B>0 \text{ 时恒成立})
	\]
	综上,自由参数 $B$ 的取值范围是 $(0, \frac{\pi}{4})$.
	
	目标函数是 $\frac{a^2+b^2}{c^2}$.利用正弦定理,将其从边长语言翻译为角度语言:
	\[
	\frac{a^2+b^2}{c^2} = \frac{\sin^2 A + \sin^2 B}{\sin^2 C}
	\]
	将所有角用参数 $B$ 代入:
	\[
	f(B) = \frac{\sin^2(\frac{\pi}{2}-2B) + \sin^2 B}{\sin^2(\frac{\pi}{2}+B)} = \frac{\cos^2(2B) + \sin^2 B}{\cos^2 B}
	\]
	问题转化为:求函数 $f(B)$ 在区间 $B \in (0, \frac{\pi}{4})$ 上的最小值.
	
	为了简化计算,我们进行变量代换.这是一个典型的换元法应用,其本质是将问题从三角函数空间映射到更易于处理的代数空间.
	令 $x = \sin^2 B$.
	由于 $B \in (0, \frac{\pi}{4})$, 新变量 $x$ 的取值范围是 $(0, \sin^2(\frac{\pi}{4})) = (0, \frac{1}{2})$.
	我们将 $f(B)$ 的表达式完全用 $x$ 来表示:
	\[
	\cos^2 B = 1-\sin^2 B = 1-x \quad \text{且} \quad \cos(2B) = 1-2\sin^2 B = 1-2x
	\]
	代入得到关于 $x$ 的函数 $g(x)$:
	\[
	g(x) = \frac{(1-2x)^2+x}{1-x} = \frac{4x^2-3x+1}{1-x}
	\]
	为求 $g(x)$ 在区间 $(0, \frac{1}{2})$ 上的最小值,我们对其求导:
	\[
	g'(x) = \frac{(8x-3)(1-x) - (4x^2-3x+1)(-1)}{(1-x)^2} = \frac{-4x^2+8x-2}{(1-x)^2}
	\]
	令 $g'(x)=0$, 即 $2x^2-4x+1 = 0$.解得 $x = 1 \pm \frac{\sqrt{2}}{2}$.
	考虑到定义域 $x \in (0, \frac{1}{2})$,我们取驻点 $x_0 = 1 - \frac{\sqrt{2}}{2}$.
	分析可知,当 $x \in (0, x_0)$ 时,$g'(x)<0$;当 $x \in (x_0, \frac{1}{2})$ 时,$g'(x)>0$.因此 $g(x)$ 在 $x=x_0$ 处取得最小值.
	将 $x_0 = 1 - \frac{\sqrt{2}}{2}$ 代入 $g(x)$.为简化计算,先对 $g(x)$ 进行代数变形:
	\[
	g(x) = \frac{-4x(1-x)+(x-1)+2}{1-x} = -4x-1+\frac{2}{1-x}
	\]
	最小值为:
	\begin{align*}
		g\left(1 - \frac{\sqrt{2}}{2}\right) &= -4\left(1-\frac{\sqrt{2}}{2}\right) - 1 + \frac{2}{1-\left(1-\frac{\sqrt{2}}{2}\right)} \\
		&= -4 + 2\sqrt{2} - 1 + \frac{2}{\frac{\sqrt{2}}{2}} \\
		&= -5 + 2\sqrt{2} + 2\sqrt{2} \\
		&= 4\sqrt{2}-5
	\end{align*}\hfill\qedsymbol
\end{solution}

\subsubsection{习题2}
\begin{solution}
	题目给出的核心约束是关于边和角的满足式条件”$a=b+2b\cos C$.为了得到我们想要的关系,我们先利用正弦定理翻译起手,目的是将边长关系翻译为角度关系.
	
	由正弦定理,设 $a=k\sin A, b=k\sin B, c=k\sin C$.代入原式得:
	\[
	k\sin A = k\sin B + 2k\sin B \cos C
	\]
	由于 $k \neq 0$, 两边同除以 $k$ 得:
	\[
	\sin A = \sin B + 2\sin B \cos C
	\]
	在 $\triangle ABC$ 中, $A = \pi-(B+C)$, 故 $\sin A = \sin(\pi-(B+C)) = \sin(B+C)$.代入上式:
	\[
	\sin(B+C) = \sin B + 2\sin B \cos C
	\]
	展开左侧:
	\[
	\sin B \cos C + \cos B \sin C = \sin B + 2\sin B \cos C
	\]
	移项整理得:
	\[
	\cos B \sin C - \sin B \cos C = \sin B
	\]
	左侧恰为差角公式,即 $\sin(C-B) = \sin B$.
	
	由于 $B,C$ 均为三角形内角,故 $B \in (0,\pi), C \in (0,\pi)$.同时 $B+C \in (0,\pi)$, 可得 $C-B \in (-\pi, \pi)$.
	由 $\sin(C-B)=\sin B$ 可得两种可能:
	\begin{itemize}
		\item $C-B = B$, 即 $C=2B$.
		\item $C-B = \pi - B$, 即 $C=\pi$.矛盾.
	\end{itemize}
	因此, $C=2B$.至此,第 (1) 问得证.
	
	第 (1) 问的结论 $C=2B$ 极大地简化了问题,它将三角形形状的自由度从 2 降至 1.我们可以选择角 $B$ 作为唯一的自由参数来描述所有满足条件的三角形.
	
	首先,我们将所有角用参数 $B$ 表示,也即所谓的选定变量,或者说是归一思想:
	\[
	C=2B, \quad A = \pi - (B+C) = \pi - 3B
	\]
	为保证 $A,B,C$ 构成一个三角形,所有内角必须为正:
	\begin{align*}
		B &> 0 \\
		C = 2B &> 0 \quad (\text{当 } B>0 \text{ 时恒成立}) \\
		A = \pi - 3B &> 0 \implies 3B < \pi \implies B < \frac{\pi}{3}
	\end{align*}
	综上,自由参数 $B$ 的取值范围是 $(0, \frac{\pi}{3})$.
	
	目标函数为 $\frac{a+c}{b}$.我们再次利用正弦定理,将其翻译为角度的表达式:
	\[
	\frac{a+c}{b} = \frac{k\sin A + k\sin C}{k\sin B} = \frac{\sin A + \sin C}{\sin B}
	\]
	将所有角用参数 $B$ 代入:
	\[
	f(B) = \frac{\sin(\pi-3B)+\sin(2B)}{\sin B}
	\]
	由于 $B \in (0, \frac{\pi}{3})$, $\sin B \neq 0$,我们可以进行化简:
	\[
	f(B) = \frac{\sin(3B)+\sin(2B)}{\sin B}
	\]
	利用和差化积公式或三倍角、二倍角公式展开.这里使用后者更为直接:
	\begin{align*}
		f(B) &= \frac{(3\sin B - 4\sin^3 B) + (2\sin B \cos B)}{\sin B} \\
		&= 3 - 4\sin^2 B + 2\cos B
	\end{align*}
	
	为了求 $f(B)$ 的范围,我们将其转化为关于单一三角函数变量的代数式.
	令 $t = \cos B$.首先确定新变量 $t$ 的范围.
	因为 $B \in (0, \frac{\pi}{3})$ 且 $y=\cos x$ 在此区间上单调递减,所以:
	\[
	t = \cos B \in \left(\cos\frac{\pi}{3}, \cos 0\right) = \left(\frac{1}{2}, 1\right)
	\]
	将原表达式中的 $\sin^2 B$ 替换为 $1-\cos^2 B = 1-t^2$:
	\begin{align*}
		g(t) &= 3 - 4(1-t^2) + 2t \\
		&= 3 - 4 + 4t^2 + 2t \\
		&= 4t^2 + 2t - 1
	\end{align*}
	问题最终转化为:求二次函数 $g(t) = 4t^2+2t-1$ 在开区间 $t \in (\frac{1}{2}, 1)$ 上的值域.
	
	这是一个开口向上的抛物线,对称轴为 $t = -\frac{2}{2 \times 4} = -\frac{1}{4}$.
	由于区间 $(\frac{1}{2}, 1)$ 完全位于对称轴的右侧,所以函数 $g(t)$ 在此区间上是严格单调递增的.
	
	因此,值域的边界由区间端点决定:
	\begin{itemize}
		\item 当 $t \to \frac{1}{2}^+$ 时, $g(t) \to 4(\frac{1}{2})^2 + 2(\frac{1}{2}) - 1 = 1+1-1=1$.
		\item 当 $t \to 1^-$ 时, $g(t) \to 4(1)^2 + 2(1) - 1 = 4+2-1=5$.
	\end{itemize}
	由于 $t$ 的取值范围是开区间,所以函数值的范围也是开区间.
	
	最终,$\frac{a+c}{b}$ 的取值范围是 $(1, 5)$.\hfill\qedsymbol
\end{solution}

\subsubsection{习题3}
\begin{solution}
	本题通过一个面积与边长的关系式来约束三角形的形状.我们的策略是利用面积公式、余弦定理来翻译这个关系式,从而确定一个内角.
	
	\textbf{1. 求角$B$.}
	一方面,三角形的面积公式为:
	\[
	S = \frac{1}{2}ac\sin B
	\]
	另一方面,题目给出的条件 $S = \frac{\sqrt{3}}{4}(a^2+c^2-b^2)$ 中,括号内的项 $a^2+c^2-b^2$ 与余弦定理结构相似.
	由余弦定理,$b^2 = a^2+c^2 - 2ac\cos B$, 可得:
	\[
	a^2+c^2-b^2 = 2ac\cos B
	\]
	将此式代入题目给定的条件中:
	\[
	S = \frac{\sqrt{3}}{4}(2ac\cos B) = \frac{\sqrt{3}}{2}ac\cos B
	\]
	于是得到了关于面积 $S$ 的两个等价表达式,令它们相等:
	\[
	\frac{1}{2}ac\sin B = \frac{\sqrt{3}}{2}ac\cos B
	\]
	由于在三角形中 $a, c$ 均为正数,可约去 $\frac{1}{2}ac$:
	\[
	\sin B = \sqrt{3}\cos B
	\]
	因为 $B \in (0, \pi)$, $\cos B=0$ 时 $\sin B \neq 0$, 故 $\cos B \neq 0$.两边同除以 $\cos B$ 得:
	\[
	\tan B = \sqrt{3}
	\]
	考虑到 $B$ 是三角形内角,解得 $B = \frac{\pi}{3}$.
	
	\textbf{2. 求解 $\triangle AOC$ 周长的取值范围}
	
	第 (1) 问确定了角 $B$ 的大小,多了一个条件,方便我们弄出第二问,来看第二问,其目标对象是 $\triangle AOC$ 的周长,我们的思路是:先将周长表达为一个函数的形式,并确定其定义域,进而求出值域.
	
	点 $O$ 是 $\triangle ABC$ 的内心,即 $AO, CO$ 分别是 $\angle A$ 和 $\angle C$ 的角平分线.因此,在 $\triangle AOC$ 中:
	\[
	\angle OAC = \frac{A}{2}, \quad \angle OCA = \frac{C}{2}
	\]
	其第三个角为:
	\[
	\angle AOC = \pi - \left(\frac{A}{2} + \frac{C}{2}\right) = \pi - \frac{A+C}{2}
	\]
	由于我们已经求出 $B=\frac{\pi}{3}$, 所以 $A+C = \pi - B = \frac{2\pi}{3}$.代入上式,我们得到一个重要的定值:
	\[
	\angle AOC = \pi - \frac{2\pi/3}{2} = \pi - \frac{\pi}{3} = \frac{2\pi}{3}
	\]
	$\triangle AOC$ 的周长 $P = AO + CO + AC$.其中 $AC = b = 2\sqrt{3}$ 是已知定值.
	
	我们选择角 $A$ 作为唯一的自由参数.在 $\triangle AOC$ 中,应用正弦定理:
	\[
	\frac{AO}{\sin(\angle OCA)} = \frac{CO}{\sin(\angle OAC)} = \frac{AC}{\sin(\angle AOC)}
	\]
	\[
	\frac{AO}{\sin(C/2)} = \frac{CO}{\sin(A/2)} = \frac{2\sqrt{3}}{\sin(2\pi/3)} = \frac{2\sqrt{3}}{\sqrt{3}/2} = 4
	\]
	由此解出 $AO = 4\sin(C/2)$ 和 $CO = 4\sin(A/2)$.
	周长表达式为:
	\[
	P(A,C) = 4\sin(A/2) + 4\sin(C/2) + 2\sqrt{3}
	\]
	将 $C = \frac{2\pi}{3}-A$ 代入,得到关于单参数 $A$ 的函数:
	\[
	P(A) = 4\left[\sin\left(\frac{A}{2}\right) + \sin\left(\frac{2\pi/3-A}{2}\right)\right] + 2\sqrt{3} = 4\left[\sin\left(\frac{A}{2}\right) + \sin\left(\frac{\pi}{3}-\frac{A}{2}\right)\right] + 2\sqrt{3}
	\]
	
	首先确定参数 $A$ 的范围.在 $\triangle ABC$ 中,各角需为正:
	\[
	A>0, \quad C = \frac{2\pi}{3}-A > 0 \implies A < \frac{2\pi}{3}
	\]
	故 $A \in (0, \frac{2\pi}{3})$.
	
	我们对括号内的三角函数部分进行化简,令其为 $g(A) = \sin(\frac{A}{2}) + \sin(\frac{\pi}{3}-\frac{A}{2})$.
	利用和差化积公式:
	\[
	g(A) = 2\sin\left(\frac{A/2 + \pi/3-A/2}{2}\right)\cos\left(\frac{A/2 - (\pi/3-A/2)}{2}\right) = 2\sin\left(\frac{\pi}{6}\right)\cos\left(\frac{A-\pi/3}{2}\right)
	\]
	由于 $2\sin(\frac{\pi}{6}) = 2 \cdot \frac{1}{2} = 1$, 所以 $g(A) = \cos\left(\frac{A}{2}-\frac{\pi}{6}\right)$.
	
	现在求 $g(A)$ 的范围.因为 $A \in (0, \frac{2\pi}{3})$:
	\[
	\frac{A}{2} \in \left(0, \frac{\pi}{3}\right) \implies \frac{A}{2}-\frac{\pi}{6} \in \left(-\frac{\pi}{6}, \frac{\pi}{6}\right)
	\]
	余弦函数 $y=\cos x$ 在区间 $(-\frac{\pi}{6}, \frac{\pi}{6})$ 上是偶函数,且在该区间上从 $\cos(-\pi/6)=\frac{\sqrt{3}}{2}$ 减到 $\cos(0)=1$ 再增回 $\cos(\pi/6)=\frac{\sqrt{3}}{2}$.其值域为 $(\cos(\frac{\pi}{6}), \cos(0)] = (\frac{\sqrt{3}}{2}, 1]$.
	所以 $g(A) \in (\frac{\sqrt{3}}{2}, 1]$.
	
	最终,周长 $P(A) = 4g(A) + 2\sqrt{3}$ 的取值范围是:
	\[
	\left(4 \cdot \frac{\sqrt{3}}{2} + 2\sqrt{3}, \quad 4 \cdot 1 + 2\sqrt{3}\right] = \left(2\sqrt{3} + 2\sqrt{3}, \quad 4 + 2\sqrt{3}\right] = \left(4\sqrt{3}, 4+2\sqrt{3}\right]
	\]
	故 $\triangle AOC$ 周长的取值范围是 $(4\sqrt{3}, 4+2\sqrt{3}]$.\hfill\qedsymbol
\end{solution}

\subsubsection{习题4}
\begin{solution}
	\textbf{1. 求角$B$.}
	
	题目给出的条件是一个混合了边与角的复杂比例式.我们的首要策略是利用正弦定理,将所有三角函数项统一替换为边长,从而将问题转化为纯粹的代数关系式,以揭示约束条件.
	
	由正弦定理,$\sin A = \frac{a}{2R}, \sin B = \frac{b}{2R}, \sin C = \frac{c}{2R}$.代入原式:
	\[
	\frac{\frac{a}{2R} - \frac{b}{2R}}{\sqrt{3}a - c} = \frac{\frac{c}{2R}}{a+b}
	\]
	消去公因式 $\frac{1}{2R}$,我们得到:
	\[
	\frac{a-b}{\sqrt{3}a - c} = \frac{c}{a+b}
	\]
	通过交叉相乘,将比例式转化为等式:
	\[
	(a-b)(a+b) = c(\sqrt{3}a - c)
	\]
	展开得:
	\[
	a^2 - b^2 = \sqrt{3}ac - c^2
	\]
	移项,整理成余弦定理的形式:
	\[
	a^2 + c^2 - b^2 = \sqrt{3}ac
	\]
	根据余弦定理,我们知道 $a^2 + c^2 - b^2 = 2ac\cos B$.于是:
	\[
	2ac\cos B = \sqrt{3}ac
	\]
	由于 $a,c$ 是三角形的边长,必为正数,可约去 $2ac$:
	\[
	\cos B = \frac{\sqrt{3}}{2}
	\]
	因为 $B$ 是三角形内角,所以 $B = \frac{\pi}{6}$.这个值满足锐角三角形中 $B \in (0, \frac{\pi}{2})$ 的要求.
	
	\textbf{2. 求解周长的取值范围}
	
	第 (1) 问固定了角 $B=\frac{\pi}{6}$,第 (2) 问又给定了边 $a=2$.此时,三角形的自由度仅剩一个.我们可以选择一个角(例如角 $A$)作为自由参数,来表达周长并确定其范围.
	
	我们选择角 $A$ 为参数.$\triangle ABC$ 是锐角三角形,因此所有角都必须在 $(0, \frac{\pi}{2})$ 内.
	\begin{itemize}
		\item $A \in (0, \frac{\pi}{2})$
		\item $B = \frac{\pi}{6} \in (0, \frac{\pi}{2})$ 
		\item $C = \pi - (A+B) = \pi - (A+\frac{\pi}{6}) = \frac{5\pi}{6} - A \in (0, \frac{\pi}{2})$
	\end{itemize}
	从 $C$ 的范围我们可以得到对 $A$ 的两个约束:
	\begin{itemize}
		\item $\frac{5\pi}{6} - A > 0 \implies A < \frac{5\pi}{6}$.此条件被 $A < \pi/2$ 包含.
		\item $\frac{5\pi}{6} - A < \frac{\pi}{2} \implies \frac{5\pi}{6} - \frac{3\pi}{6} < A \implies A > \frac{2\pi}{6} = \frac{\pi}{3}$.
	\end{itemize}
	综上,参数 $A$ 的取值范围是 $(\frac{\pi}{3}, \frac{\pi}{2})$.
	
	确定完后,我们看看周长 $P = a+b+c = 2+b+c$,利用正弦定理将 $b,c$ 用参数 $A$ 表示.
	\[
	\frac{a}{\sin A} = \frac{b}{\sin B} = \frac{c}{\sin C} \implies \frac{2}{\sin A} = \frac{b}{\sin(\pi/6)} = \frac{c}{\sin(5\pi/6 - A)}
	\]
	解得:
	\[
	b = \frac{2\sin(\pi/6)}{\sin A} = \frac{1}{\sin A}
	\]
	\[
	c = \frac{2\sin(5\pi/6 - A)}{\sin A}
	\]
	所以,周长 $P$ 是关于 $A$ 的函数:
	\begin{align*}
		P(A) &= 2 + \frac{1}{\sin A} + \frac{2\sin(5\pi/6 - A)}{\sin A} \\
		&= 2 + \frac{1 + 2\left(\sin\frac{5\pi}{6}\cos A - \cos\frac{5\pi}{6}\sin A\right)}{\sin A} \\
		&= 2 + \frac{1 + 2\left(\frac{1}{2}\cos A + \frac{\sqrt{3}}{2}\sin A\right)}{\sin A} \\
		&= 2 + \frac{1 + \cos A + \sqrt{3}\sin A}{\sin A} \\
		&= 2 + \frac{1+\cos A}{\sin A} + \sqrt{3}
	\end{align*}
	
	化简,然后分析函数 $g(A) = \frac{1+\cos A}{\sin A}$.利用半角公式,简单期间,设$\cot A/2 = \frac{1}{\tan A/2}$,这里用到余切函数了,当然你换个元也是一样的效果.
	\[
	g(A) = \frac{2\cos^2(A/2)}{2\sin(A/2)\cos(A/2)} = \cot(A/2)
	\]
	因此,周长函数为 $P(A) = 2 + \sqrt{3} + \cot(A/2)$.
	
	注意道当 $A \in (\frac{\pi}{3}, \frac{\pi}{2})$ 时,
	\[
	A \in \left(\frac{\pi}{3}, \frac{\pi}{2}\right) \implies \frac{A}{2} \in \left(\frac{\pi}{6}, \frac{\pi}{4}\right)
	\]
	函数 $y = \cot x$ 在区间 $(\frac{\pi}{6}, \frac{\pi}{4})$ 上是单减.
	因此,函数的值域由区间端点决定:
	\begin{itemize}
		\item 当 $\frac{A}{2} \to \frac{\pi}{6}^+$ 时,$\cot(\frac{A}{2}) \to \cot(\frac{\pi}{6}) = \sqrt{3}$.
		\item 当 $\frac{A}{2} \to \frac{\pi}{4}^-$ 时,$\cot(\frac{A}{2}) \to \cot(\frac{\pi}{4}) = 1$.
	\end{itemize}
	所以,$\cot(A/2)$ 的取值范围是 $(1, \sqrt{3})$.
	
	综上,周长 $P(A) = 2+\sqrt{3} + \cot(A/2)$ 的取值范围是:
	\[
	\left(2+\sqrt{3} + 1, \quad 2+\sqrt{3} + \sqrt{3}\right) = \left(3+\sqrt{3}, 2+2\sqrt{3}\right)
	\]
	故 $\triangle ABC$ 周长的取值范围是 $(3+\sqrt{3}, 2+2\sqrt{3})$.\hfill\qedsymbol
\end{solution}

\subsection{面积问题}
\subsubsection{习题5}
\begin{solution}
	\textbf{1. 求 $\angle ADC$}
	
	本题要我们求 $\triangle ADC$.题目直接给出了这个三角形的三条边长:$AD=5, DC=3, AC=7$.当一个三角形的三边确定时,其形状和所有内角也随之完全确定.
	
	先应用余弦定理来求解 $\angle ADC$.在 $\triangle ADC$ 中:
	\[
	AC^2 = AD^2 + DC^2 - 2(AD)(DC)\cos(\angle ADC)
	\]
	代入已知数据:
	\[
	7^2 = 5^2 + 3^2 - 2(5)(3)\cos(\angle ADC)
	\]
	\[
	49 = 25 + 9 - 30\cos(\angle ADC)
	\]
	\[
	49 = 34 - 30\cos(\angle ADC)
	\]
	整理得到:
	\[
	15 = -30\cos(\angle ADC) \implies \cos(\angle ADC) = -\frac{1}{2}
	\]
	由于 $\angle ADC$ 是三角形的内角,其范围为 $(0, \pi)$,因此 $\angle ADC = \frac{2\pi}{3}$.
	
	\textbf{2. 求 $\triangle ABC$ 面积}
	
	为求 $\triangle ABC$ 的面积,我们需要确定更多的边或角.关键在于利用连接两个三角形的条件 $\angle DAC = \angle ABC$.
	
	令 $\angle ABC = B$, 则 $\angle DAC = B$.
	在 $\triangle ADC$ 中,我们已知三边和一角,可以利用正弦定理求出 $\sin B$:
	\[
	\frac{AC}{\sin(\angle ADC)} = \frac{DC}{\sin(\angle DAC)}
	\]
	\[
	\frac{7}{\sin(2\pi/3)} = \frac{3}{\sin B} \implies \sin B = \frac{3\sin(2\pi/3)}{7} = \frac{3(\sqrt{3}/2)}{7} = \frac{3\sqrt{3}}{14}
	\]
	
	现在我们的焦点转移到 $\triangle ABD$.
	我们知道 $\angle ADC = \frac{2\pi}{3}$, 故其补角 $\angle ADB = \pi - \frac{2\pi}{3} = \frac{\pi}{3}$.
	在 $\triangle ABD$ 中,我们已知:
	\begin{itemize}
		\item 边 $AD = 5$
		\item 角 $\angle ABD = B$
		\item 角 $\angle ADB = \frac{\pi}{3}$
	\end{itemize}
	第三个角为 $\angle BAD = \pi - B - \frac{\pi}{3} = \frac{2\pi}{3}-B$.
	应用正弦定理于 $\triangle ABD$:
	\[
	\frac{AB}{\sin(\angle ADB)} = \frac{BD}{\sin(\angle BAD)} = \frac{AD}{\sin(\angle ABD)}
	\]
	\[
	\frac{AB}{\sin(\pi/3)} = \frac{BD}{\sin(2\pi/3 - B)} = \frac{5}{\sin B}
	\]
	由此可解得边长 $BD$:
	\[
	BD = \frac{5\sin(2\pi/3 - B)}{\sin B}
	\]
	为计算此值,我们需要 $\cos B$.由于 $\triangle ABC$ 中 $D$ 在 $BC$ 边上,则 $C = \angle ACD$ 必为锐角(否则 $C+\angle ADC \ge \pi$),故 $\angle DAC = B$ 也必为锐角.因此 $\cos B > 0$.
	\[
	\cos B = \sqrt{1-\sin^2 B} = \sqrt{1 - \left(\frac{3\sqrt{3}}{14}\right)^2} = \sqrt{1-\frac{27}{196}} = \sqrt{\frac{169}{196}} = \frac{13}{14}
	\]
	展开 $\sin(2\pi/3 - B)$:
	\begin{align*}
		BD &= \frac{5(\sin(2\pi/3)\cos B - \cos(2\pi/3)\sin B)}{\sin B} \\
		&= \frac{5((\sqrt{3}/2)(13/14) - (-1/2)(3\sqrt{3}/14))}{3\sqrt{3}/14} \\
		&= \frac{5(13\sqrt{3}/28 + 3\sqrt{3}/28)}{3\sqrt{3}/14} = \frac{5(16\sqrt{3}/28)}{3\sqrt{3}/14} = \frac{5(4\sqrt{3}/7)}{3\sqrt{3}/14} = \frac{20\sqrt{3}}{7} \cdot \frac{14}{3\sqrt{3}} = \frac{40}{3}
	\end{align*}
	
	接下来,计算面积即可,$\triangle ABC$ 的面积等于 $\triangle ABD$ 与 $\triangle ADC$ 的面积之和.
	\[
	S_{\triangle ADC} = \frac{1}{2}AD \cdot DC \sin(\angle ADC) = \frac{1}{2}(5)(3)\sin\left(\frac{2\pi}{3}\right) = \frac{15}{2} \cdot \frac{\sqrt{3}}{2} = \frac{15\sqrt{3}}{4}
	\]
	\[
	S_{\triangle ABD} = \frac{1}{2}AD \cdot BD \sin(\angle ADB) = \frac{1}{2}(5)\left(\frac{40}{3}\right)\sin\left(\frac{\pi}{3}\right) = \frac{100}{3} \cdot \frac{\sqrt{3}}{2} = \frac{50\sqrt{3}}{3}
	\]
	总面积为:
	\[
	S_{\triangle ABC} = S_{\triangle ADC} + S_{\triangle ABD} = \frac{15\sqrt{3}}{4} + \frac{50\sqrt{3}}{3} = \sqrt{3}\left(\frac{45+200}{12}\right) = \frac{245\sqrt{3}}{12}
	\]
	故 $\triangle ABC$ 的面积为 $\frac{245\sqrt{3}}{12}$.\hfill\qedsymbol
\end{solution}